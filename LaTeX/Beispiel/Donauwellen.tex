\newpage

\begin{recipe}
[ % Optionale Eingaben
    bakingtime={\unit[30]{min}},
    bakingtemperature={\unit[200]{\textcelcius}},
%    portion = {\portion{5-6}}
]
{Donauwellen}
    
    \graph
    {% Bilder
%        small=glass,    % kleines Bild
        big=./Bilder/Donauwellen % gro�es (l�ngeres) Bild
    }
    
    \ingredients
    {% Zutaten
	    \unit[250]{g} & Butter \\
		\unit[350]{g} & Mehl \\
		6 & Eier\\
		\unit[200]{g} & Zucker\\
		\unit[1]{Pck.} & Vanillezucker \\
		\unit[1/2]{Pck.} & Backpulver \\
		2 Gl�ser & Sauerkirschen\\
		\unit[2]{EL} & Kakao\\
		\ \\
		\unit[1]{Pck.} & Paradiescreme Vanille \\
		\unit[400]{ml} & Schlagsahne \\
		\ \\
		\unit[1-2]{Pck.}& Schokoglasur \\
    }
    
        
    
    
    \preparation
    { % Zubereitung
        \step Butter schaumig r�hren. Nach und Nach Zucker und Eier zugeben. Mehl und Backpulver mischen und essl�ffelweise zugeben.
        \step Etwa die H�lfte des Teiges auf das Backblech streichen. Den restlichen Teig mit dem Kakao verr�hren und auf dem hellen Teig verteilen.
        \step Mit einer Gabel Wellen im Teig ziehen. Die abgetropften Kirschen auf dem Teig verteilen. Bei \unit[200]{\textcelcius} f�r \unit[30]{min} backen.
        \step Die Paradiescreme mit der Sahne f�r ungef�hr \unit[3]{min} mit dem Handr�hrger�t schlagen und auf dem erkalteten Kuchen verteilen. Mit der Schokoglasur verzieren.
    }
    
    \hint
    {% Tipp
        Eine Portion Paradiescreme ist wenig, zwei sind viel.
    }

\end{recipe}
