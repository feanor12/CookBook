\documentclass[%
a4paper,
%twoside,
11pt
]{article} 

% encoding, font, language
\usepackage[T1]{fontenc}
\usepackage[latin1]{inputenc}
\usepackage{lmodern}
\usepackage[ngerman]{babel}
\usepackage{graphicx}
\usepackage{nicefrac}


\usepackage[
    nowarnings,
    %myconfig 
]
{xcookybooky}
 

\DeclareRobustCommand{\textcelcius}{\ensuremath{^{\circ}\mathrm{C}}}


\setcounter{secnumdepth}{1}
\renewcommand*{\recipesection}[2][]
{%
    \subsection[#1]{#2}
}
\renewcommand{\subsectionmark}[1]
{% no implementation to display the section name instead
}
%\setRecipeColors{recipename=blue}

\usepackage{hyperref}    % must be the last package
\hypersetup{%
%    pdfauthor            = {Sven Harder},
    pdftitle             = {Mamas Kochbuch},
    pdfsubject           = {Recipes},
    pdfkeywords          = {example, recipes, cookbook, xcookybooky},
    pdfstartview         = {FitV},
    pdfview              = {FitH},
    pdfpagemode          = {UseNone}, % Options; UseNone, UseOutlines
    bookmarksopen        = {true},
    pdfpagetransition    = {Glitter},
    colorlinks           = {true},
    linkcolor            = {black},
    urlcolor             = {blue},
    citecolor            = {black},
    filecolor            = {black},
}

\hbadness=10000 % Ignore underfull boxes

\begin{document}

\title{Mamas Kochbuch}

\maketitle

\setHeadlines
{% translation
    inghead = Zutaten,
    prephead = Zubereitung,
    hinthead = Tipp,
    continuationhead = Fortsetzung,
    continuationfoot = Fortsetzung auf n\"achster Seite,
    portionvalue = Personen,
}


\tableofcontents

\vspace{5em}
\newpage
\section{+Kategorie}  

\begin{center}
\includegraphics[width=0.8\textwidth]{+Kategoriefoto}
\end{center}

% % % Hier kommen die +Kategorie-Rezepte rein % % %
\newpage

\begin{recipe}
[ % Optionale Eingaben
    bakingtime={+Backzeit},
    bakingtemperature={+Temperatur \textcelcius}, % �C nicht �bergeben
%    portion = {+Portionen} % Anzahl der Portionen noch nicht implementiert
]
{+Name}
    
    \graph
    {% Bilder        
    
	    big=+Foto1 		% gro�es (l�ngeres) Bild
%        small=+Foto2,    % kleines Bild noch nicht implementiert

    }
    
    \ingredients
    {% Zutaten
	    +Menge & +Zutat \\
		\ \\ 					% Leerzeile zur Trennung (z.B. Boden u. Creme)
		+Menge & +Zutat \\
    }
    
        
    
    
    \preparation
    { % Zubereitung
        \step +Beschreibung % Durch Zeilenumbruch getrennt?
        \step +Beschreibung 
    }
    
%    \hint
%    {% Tipp
%        +Tipp  % Tipp noch nicht implementiert
%    }

\end{recipe}

%\newpage

\end{document} 